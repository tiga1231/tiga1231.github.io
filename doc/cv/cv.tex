%%%%%%%%%%%%%%%%% PREAMBLE %%%%%%%%%%%%%%%%%%%%%%%%%%%%
%Change the font size of your document - 10pt, 12.1pt, etc.
\documentclass[letterpaper,11pt,oneside]{article}
\usepackage[utf8]{inputenc}
\usepackage{setspace}
\usepackage{hyperref}

\usepackage{graphicx}
\graphicspath{ {images/}} %upload your signature to this file
%Change the margins to fit your CV/resume content
\usepackage[left=1in, right=1in, bottom=1.25in, top=1.25in]{geometry}

%Skype information - include your Skype name for a link to add you on Skype
% \newcommand*{\Skype}{\href{skype:john.smith?add}{john.smith}} 
% \newcommand{\Absender}[1][\normalsize]{\Skype} 

%Changes the page numbers - {arabic}=arabic numerals, {gobble}=no page numbers, {roman}=Roman numerals
\pagenumbering{gobble}

%%%%%%%%%%%%%%%%% END OF PREAMBLE %%%%%%%%%%%%%%%%%%%%%

\begin{document}

%%%%%%%%%%%%%%%%% NAME OF APPLICANT %%%%%%%%%%%%%%%%%%%

\noindent  \LARGE{\textbf{Mingwei Li}}  \\
\vspace{-2ex}
% \hline 
\normalsize

%%%%%%%%%%%%%%%%% CONTACT INFORMATION %%%%%%%%%%%%%%%%%
% Your email address, website, and Skype name are links to send email, open your website and add you on Skype. 

\begin{center}
\begin{tabular}{l r}
Vanderbilt University & \hspace{3in} \href{https://tiga1231.github.io/}{Website} \\
Department of Computer Science & \href{https://scholar.google.com/citations?user=CHuXIuIAAAAJ}{Google Scholar} \\
                                    & mingwei.li at vanderbilt.edu \\
                                    & Jobs: mwli93021 at gmail
\end{tabular}
\end{center}

\vspace{1em}

%%%%%%%%%%%%%%%%% MAIN BODY %%%%%%%%%%%%%%%%%%%%%%%%%%%
% The main body is contained in a tabular environment. To move sections onto the next page, simply end the tabular environment and begin a new tabular environment.

\noindent \begin{tabular}{@{} r l}
\Large{Experience}    & \textbf{Vanderbilt University} \\
    & Postdoctoral Scholar, Research, 2021 - Present. \\
    & \\
% \end{tabular}
% \noindent \begin{tabular}{@{} l l}
\Large{Education}    & \textbf{University of Arizona} \\

    & Doctor of Philosophy in Computer Science, 2016 - Aug 21, 2021. \\
    & Field: Data Visualization \\
    & Minor: Mathematics \\
    & Thesis: Algebraic Visual Design for Deep Learning \\
    & Advisor: Prof. Carlos Scheidegger \\
    % & GPA: 4.0/4.0 \\
    & \\
    & \textbf{Hong Kong University of Science and Technology} \\
    & Bachelor in Electronic Engineering, Honor Research Program, 2011 - 2015 \\
    & Minor: Mathematics \\
    & Thesis: Wi-Fi based Indoor Localization \\
    & Advisor: Prof. Shenghui Song \\
    % & GPA: 3.682/4.3 \\
    & \\

\Large{Teaching}   & \textbf{Department of Computer Science, University of Arizona} \\
    & Teaching Assistant, CSC 245, Introduction to Discrete Structures, Summer 2018 \\
    & Teaching Assistant, CSC 337, Web Programming, Fall 2016 \\
    &\textbf{Department of Electronic and Computer Engineering, HKUST} \\
    & Student Helper, ELEC 1100, Introduction of Robotics, Fall 2012 \\
    & \\

\Large{Awards and}
    & \textbf{GPSC Travel Grant} \\
\Large{Fellowships} 
    & University of Arizona, Oct 2018 \\
    & \textbf{Graduate Assistantship, Department of Computer Science} \\
    & University of Arizona, 2016-2021 \\
    & \textbf{Dean’s List, School of Engineering} \\
    & Hong Kong University of Science and Technology, 2011-2014 \\
    & \textbf{Scholarship for Continuing Undergraduate Students} \\
    & Hong Kong University of Science and Technology, 2011-2014 \\
    & \\

\Large{Tools and Skills}
    & JavaScript (D3.js, WebGL, React.js) \\
    & Python (PyTorch, Flask, Matplotlib) \\
    & Markdown, HTML\&CSS, \LaTeX \\
    & Lua (LÖVE, LÖVR, Neovim) \\ 
    & Linux, git, vim \\
    & Blender

\end{tabular}

% \noindent \begin{tabular}{p{0.21\linewidth} p{0.7\linewidth}}
\pagebreak

% \Large{Selected Work}
\section*{Selected Works}

\subsection*{Thesis, 2021}
    \begin{itemize}
        \item \textbf{Algebraic Visual Design for Deep Learning} Mingwei Li. \url{https://repository.arizona.edu/handle/10150/661598}
    \end{itemize}
\subsection*{Deep Learning Visualization, Multi-dimensional Data, 2017-Current}
    \begin{itemize}
        \item \textbf{Neuralcubes: Deep representations for visual data exploration}. 
            Zhe Wang, Dylan Cashman, Mingwei Li, Jixian Li, Matthew Berger, Joshua A Levine, Remco Chang, Carlos Scheidegger. 
            2021 IEEE International Conference on Big Data (Big Data), 550-561
        \item \textbf{UnProjection: Leveraging Inverse-Projections for Visual Analytics of High Dimensional Data.}
            Mateus Espadoto, Gabriel Appleby, Ashley Suh, Dylan Cashman, Mingwei Li, Carlos E Scheidegger, Erik Wesley Anderson, Remco Chang, Alexandru Cristian Telea. 
            IEEE Transactions on Visualization and Computer Graphics, 2021
        \item \textbf{[Best Submission Award] Toward Comparing DNNs with UMAP Tour. }
            Mingwei Li, and Carlos Scheidegger. 
            VISxAI workshop, IEEE VIS 2020. Available online \url{https://tiga1231.github.io/umap-tour/}
        \item \textbf{Visualizing Neural Networks with the Grand Tour}
            Mingwei Li, Zhenge Zhao, and Carlos Scheidegger.
            Distill.pub, 2020. Available at \url{https://distill.pub/2020/grand-tour/}
        \item \textbf{Visualizing Neuron Activations with the Grand Tour}
            Mingwei Li, Zhenge Zhao, Carlos Scheidegger. 
            Proceedings of the Workshop on Visualization for AI (VISxAI), 2018.
    \end{itemize}

\subsection*{Graphical Perceptions, User Studies, Algebraic Visualization, 2018-Current}
    \begin{itemize}
        \item \textbf{Looks Good to Me: Visualizations as Sanity Checks}
            M. Correll, M. Li, G. Kindlmann, and C. Scheidegger. 
            IEEE Transactions in Visualization and Computer Graphics (Proceedings of InfoVis), 2018.
    \end{itemize}

\subsection*{Graph Drawing, 2020-Current}
\begin{itemize}
    \item \textbf{Visualizing Evolving Trees.}
        K Gray, M Li, R Ahmed, S Kobourov. 
        arXiv preprint arXiv:2106.08843, 2022
    \item \textbf{Multicriteria Scalable Graph Drawing via Stochastic Gradient Descent, $(SGD)^2$.}
        R Ahmed, F De Luca, S Devkota, S Kobourov, M Li. 
        IEEE Transactions on Visualization and Computer Graphics 28 (6), 2388-2399, 2021
    \item \textbf{[Best Paper Award] Graph Drawing via Gradient Descent, $(GD)^2$. }
        Ahmed R, De Luca F, Devkota S, Kobourov S, Li M. 
        arXiv preprint arXiv:2008.05584. 2020 Aug 12. Demo: \url{http://hdc.cs.arizona.edu/~mwli/graph-drawing/}\\
\end{itemize}



\end{document}

