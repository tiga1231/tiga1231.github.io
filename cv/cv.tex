%%%%%%%%%%%%%%%%% PREAMBLE %%%%%%%%%%%%%%%%%%%%%%%%%%%%
%Change the font size of your document - 10pt, 12.1pt, etc.
\documentclass[letterpaper,11pt,oneside]{article}
\usepackage[utf8]{inputenc}
\usepackage{setspace}
\usepackage{hyperref}

\usepackage{graphicx}
\graphicspath{ {images/}} %upload your signature to this file
%Change the margins to fit your CV/resume content
\usepackage[left=1in, right=1in, bottom=1.25in, top=1.25in]{geometry}

%Skype information - include your Skype name for a link to add you on Skype
% \newcommand*{\Skype}{\href{skype:john.smith?add}{john.smith}} 
% \newcommand{\Absender}[1][\normalsize]{\Skype} 

%Changes the page numbers - {arabic}=arabic numerals, {gobble}=no page numbers, {roman}=Roman numerals
\pagenumbering{gobble}

%%%%%%%%%%%%%%%%% END OF PREAMBLE %%%%%%%%%%%%%%%%%%%%%

\begin{document}

%%%%%%%%%%%%%%%%% NAME OF APPLICANT %%%%%%%%%%%%%%%%%%%

\noindent  \LARGE{\textbf{Mingwei Li}}  \\
\vspace{-3ex}
% \hline 
\normalsize

%%%%%%%%%%%%%%%%% CONTACT INFORMATION %%%%%%%%%%%%%%%%%
% Your email address, website, and Skype name are links to send email, open your website and add you on Skype. 

\begin{center}
\begin{tabular}{l r}
Vanderbilt University & \hspace{3in} \href{https://tiga1231.github.io/}{Website}, \href{https://scholar.google.com/citations?user=CHuXIuIAAAAJ}{Google Scholar}\\
Department of Computer Science 
    & mingwei.li at vanderbilt.edu \\
    & Jobs: mwli93021 at gmail
\end{tabular}
\end{center}

\vspace{1em}

%%%%%%%%%%%%%%%%% MAIN BODY %%%%%%%%%%%%%%%%%%%%%%%%%%%
% The main body is contained in a tabular environment. To move sections onto the next page, simply end the tabular environment and begin a new tabular environment.

\noindent \begin{tabular}{@{} r l}
\Large{Experience}    & \textbf{Vanderbilt University} \\
    & Postdoctoral Scholar, Research, 2021 - Present. \\
    & \\
% \end{tabular}
% \noindent \begin{tabular}{@{} l l}
\Large{Education}    & \textbf{University of Arizona} \\

    & Doctor of Philosophy in Computer Science, 2016 - Aug 21, 2021. \\
    & Field: Data Visualization; Minor in Mathematics \\
    & Thesis: Algebraic Visual Design for Deep Learning \\
    & Advisor: Prof. Carlos Scheidegger \\
    % & GPA: 4.0/4.0 \\
    & \textbf{Hong Kong University of Science and Technology} \\
    & Bachelor in Electronic Engineering, Honor Research Program, 2011 - 2015 \\
    & Minor: Mathematics \\
    & Thesis: Wi-Fi based Indoor Localization \\
    & Advisor: Prof. Shenghui Song \\
    % & GPA: 3.682/4.3 \\
    & \\

\Large{Teaching}   & \textbf{Department of Computer Science, University of Arizona} \\
    & Teaching Assistant, CSC 245, Introduction to Discrete Structures, Summer 2018 \\
    & Teaching Assistant, CSC 337, Web Programming, Fall 2016 \\
    &\textbf{Department of Electronic and Computer Engineering, HKUST} \\
    & Student Helper, ELEC 1100, Introduction of Robotics, Fall 2012 \\
    & \\

\Large{Awards and}
    & \textbf{GPSC Travel Grant} \\
\Large{Fellowships} 
    & University of Arizona, Oct 2018 \\
    & \textbf{Graduate Assistantship, Department of Computer Science} \\
    & University of Arizona, 2016-2021 \\
    & \textbf{Dean’s List, School of Engineering} \\
    & Hong Kong University of Science and Technology, 2011-2014 \\
    & \textbf{Scholarship for Continuing Undergraduate Students} \\
    & Hong Kong University of Science and Technology, 2011-2014 \\
    \\
    
\Large{Service}
    %% https://www.ieeevis.org/year/2022/info/papers-sessions
    & \textbf{External Reviewer} TVCG and IEEE VIS 2018-current \\
    & \textbf{Session Chair} Short Papers: Visual Analytics, Decision Support, \\
    & and Machine Learning, IEEE VIS 2022\\
    
    \\
\Large{Tools and Skills}
    & Python (PyTorch, Numpy, Flask, Matplotlib) \\
    & JavaScript (D3.js, WebGL) \\
    & Linux, Git, Vim \\
    & Markdown, HTML\&CSS, \LaTeX \\
    & C++ (PyTorch), Lua (LÖVE, LÖVR, Neovim) \\ 
    % & Blender

\end{tabular}

% \noindent \begin{tabular}{p{0.21\linewidth} p{0.7\linewidth}}
\pagebreak

\section*{Selected Works}

\subsection*{Thesis, 2021}
I discussed the algebraic structures involved in designing visualizations for making sense of deep neural networks.
    \begin{itemize}
        \item \textbf{Algebraic Visual Design for Deep Learning} Mingwei Li. \url{https://repository.arizona.edu/handle/10150/661598}
    \end{itemize}

\subsection*{Deep Learning Visualization, High-dimensional Data, 2017-Current}
We study the intersection of deep learning and data visualization. 
    We use visualization techniques, such as Grand Tour, to make sense of the internal working of neural networks. 
    We harness the power of universal learner for visualization designs and practices, such as understanding dimensionality reduction plots or speeding up data summary in big data visualizations.
    \begin{itemize}
        \item \textbf{[Best Submission Award] Toward Comparing DNNs with UMAP Tour. }
            Mingwei Li, and Carlos Scheidegger. 
            VISxAI workshop, IEEE VIS 2020. Available online \url{https://tiga1231.github.io/umap-tour/}
        \item \textbf{Visualizing Neural Networks with the Grand Tour}
            Mingwei Li, Zhenge Zhao, and Carlos Scheidegger.
            Distill.pub, 2020. Available at \url{https://distill.pub/2020/grand-tour/}
        \item \textbf{Neuralcubes: Deep representations for visual data exploration}. 
            Zhe Wang, Dylan Cashman, Mingwei Li, Jixian Li, Matthew Berger, Joshua A Levine, Remco Chang, Carlos Scheidegger. 
            2021 IEEE International Conference on Big Data (Big Data), 550-561
        \item \textbf{UnProjection: Leveraging Inverse-Projections for Visual Analytics of High Dimensional Data.}
            Mateus Espadoto, Gabriel Appleby, Ashley Suh, Dylan Cashman, Mingwei Li, Carlos E Scheidegger, Erik Wesley Anderson, Remco Chang, Alexandru Cristian Telea. 
            IEEE Transactions on Visualization and Computer Graphics (TVCG), 2021
        \item \textbf{ConceptLens: Visually Analyzing the Consistency of Semantic Manipulation in GANs}
            Sangwon Jeong, Mingwei Li, Matthew Berger, Shusen Liu.
            IEEE VIS 2023 Short Paper.


    \end{itemize}


\subsection*{Graph Drawing, 2020-Current}
We optimized node placements for graph visualizations in node-link diagrams. 
We optimized multiple readability criteria (e.g. minimize number of edge crossings, preserve node neighborhoods) using gradient-based or force-directed methods. 

\begin{itemize}
    \item \textbf{[Best Paper Award] Graph Drawing via Gradient Descent, $(GD)^2$. }
        Ahmed R, De Luca F, Devkota S, Kobourov S, Li M. 
        arXiv preprint arXiv:2008.05584. 2020 Aug 12. Demo: \url{http://hdc.cs.arizona.edu/~mwli/graph-drawing/}
    \item \textbf{Multicriteria Scalable Graph Drawing via Stochastic Gradient Descent, $(SGD)^2$.}
        R Ahmed, F De Luca, S Devkota, S Kobourov, M Li. 
        IEEE Transactions on Visualization and Computer Graphics 28 (6), 2388-2399, 2021
    \item \textbf{Visualizing Evolving Trees}
        Kathryn Gray, Mingwei Li, Reyan Ahmed, and Stephen Kobourov. 
        Graph Drawing and Network Visualization: 30th International Symposium, GD 2022, Tokyo, Japan, September 13–16, 2022.
\end{itemize}


\subsection*{Graphical Perceptions, User Studies, Algebraic Visualization, 2018-Current}
We studied how human (mis-)read various types of visualization designs when reading explanations of deep learning models, or when data have certain flaws.
    \begin{itemize}
        \item \textbf{Graphical Perception of Saliency-based Model Explanations}
            Yayan Zhao, Mingwei Li, and Matthew Berger.
            Proceedings of the 2023 CHI Conference on Human Factors in Computing Systems (CHI '23)
        \item \textbf{Looks Good to Me: Visualizations as Sanity Checks}
            M. Correll, M. Li, G. Kindlmann, and C. Scheidegger. 
            IEEE Transactions in Visualization and Computer Graphics (Proceedings of InfoVis), 2018.
    \end{itemize}


\end{document}

